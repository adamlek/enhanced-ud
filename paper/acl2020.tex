%
% File acl2020.tex
%
%% Based on the style files for ACL 2020, which were
%% Based on the style files for ACL 2018, NAACL 2018/19, which were
%% Based on the style files for ACL-2015, with some improvements
%%  taken from the NAACL-2016 style
%% Based on the style files for ACL-2014, which were, in turn,
%% based on ACL-2013, ACL-2012, ACL-2011, ACL-2010, ACL-IJCNLP-2009,
%% EACL-2009, IJCNLP-2008...
%% Based on the style files for EACL 2006 by 
%%e.agirre@ehu.es or Sergi.Balari@uab.es
%% and that of ACL 08 by Joakim Nivre and Noah Smith

\documentclass[11pt,a4paper]{article}
\usepackage[hyperref]{acl2020}
\usepackage{times}
\usepackage{latexsym}
\renewcommand{\UrlFont}{\ttfamily\small}

% This is not strictly necessary, and may be commented out,
% but it will improve the layout of the manuscript,
% and will typically save some space.
\usepackage{microtype}

%\aclfinalcopy % Uncomment this line for the final submission
%\def\aclpaperid{***} %  Enter the acl Paper ID here

%\setlength\titlebox{5cm}
% You can expand the titlebox if you need extra space
% to show all the authors. Please do not make the titlebox
% smaller than 5cm (the original size); we will check this
% in the camera-ready version and ask you to change it back.

\usepackage{subcaption}
\usepackage{cleveref}
\usepackage{tikz}
\usetikzlibrary{arrows,positioning,fit,shapes}
\usepackage{todonotes}

\newcommand\BibTeX{B\textsc{ib}\TeX}

\title{How much of enhanced UD is contained in UD?}
% How enhanced is enhanced UD?

\author{First Author \\
  Affiliation / Address line 1 \\
  Affiliation / Address line 2 \\
  Affiliation / Address line 3 \\
  \texttt{email@domain} \\\And
  Second Author \\
  Affiliation / Address line 1 \\
  Affiliation / Address line 2 \\
  Affiliation / Address line 3 \\
  \texttt{email@domain} \\}

\date{}

\begin{document}
\maketitle
\begin{abstract}
  TODO
\end{abstract}

\section{Introduction}
(ADAM)

In this paper we present an approach to producing enhanced universal
dependencies.

General idea: as described by the website
(https://universaldependencies.org/u/overview/enhanced-syntax.html)
for many cases, enhanced UD is a function of UD.

Hypothesis: most information about EUD is already contained in UD.

Describe task again.

What is ELAS and EULAS  score.

Implement the function for the common case, and see how we fare on the
ELAS and EULAS  score.

Results in 2 lines:
- scores
- efficient
- transparent

\section{Method}


In essence, our method is to apply the recipes provided by TODO to
transform UD \emph{trees} into EUD graphs.

To do so, we use a tree-matching procedure against a UD tree, and
locally insert edges corresponding to enhanced dependencies (and
sometimes delete unwanted edges). As a particular case, we may
re-label some edges. Once this is done, we convert the result back to
(enhanced) CONLL \todo{citation} format.

Perhaps surprisingly, the patterns that we need to recognize are
simple, involving only three nodes. The two patterns to recognised are
shown in \cref{fig:patterns}. Essentially, we need to match on three
connected nodes.  We need to identify types of paterns. First, two
arcs forming a two-step path
(\cref{fig:pat-nsubj-conj,fig:pat-obl,fig:pat-rel}). We refer to this
style pattern as ``Type 1''.  Second, with two arcs pointing away from
a central node (\cref{fig:pat-aux-conj,fig:pat-xcomp}), refered to as
``Type 2''.

\begin{itemize}
\item Type 1 pattern with a relation label, which can be any of
  ``nsubj",``obj", ``amod", ``advcl",``obl", ``mark", ``nmod", followed by a
  ``conj'' label. (\cref{fig:pat-nsubj-conj}.) In this case we add an
  edge with the relation label to the other conjunct.
\item Type 2 pattern with a a relation label being either ``nsubj'' or
  ``aux'', and a ``conj'' label. (\cref{fig:pat-aux-conj}.) We add a relation a relation label to
  the other conjunct, but only if the conjunct is not itself
  ``nsubj''. Indeed, if it were, then we are conjoining two full
  sentences and then there is no need for an enhanced dependency.
\item Type 2 pattern with ``xcomp'' and ``nsubj''. Here we add an
  "nsubj:xsubj" edge. (\cref{fig:pat-xcomp}.)
\item Type 1 pattern, with ``acl:relcl" followed by a relation label
  which can be either ``nsubj",``obj",``obl",
  "advmod". (\cref{fig:pat-rel}.) The target node should also be a
  \emph{relative} pronoun, ie. it POS is ``PRON'' and its XPOS either
  ``WP'' (who, whom) or ``WDT'' (that, which). Indeed, this pattern is
  also found with other type of pronouns, but then it does not
  correspond to a relative clause.  In this case we add a ref edge to
  the pronoun and a (reverse) relation edge between the first and
  second node. The original relation edge is deleted.\todo{check}
\item Type 1 pattern, with a conjunction followed by a case
  marking. (\cref{fig:pat-obl}.) Exhaustively, the type of labels are ``case" followed by
  ``obl" or ``nmod"; ``cc" followed by ``conj", ``mark" followed by ``advcl"
  or ``acl". In this case we enhance the label with the lemma of the target node.
\end{itemize}
In all cases, we have additional constraints on the
(edge) labels. For Type 1, the relation can be either nsubj or aux


(ADAM) Insert figure:

``Paul and Mary eat.''
``She was reading or watching something''
\tikzstyle{word}=[ellipse,draw=blue!50,fill=blue!20,thick]
\tikzstyle{newedge}=[very thick]
\begin{figure}
\begin{subfigure}{\columnwidth}
  \centering
  \begin{tikzpicture}[inner sep=1mm]
    \node[word] (eat) {Eat};
    \node[word] (paul) [right=of eat] {Paul};
    \node[word] (mary) [right=of paul] {Mary};
    \draw[->] (eat) -- node[above] {nsubj} (paul);
    \draw[->] (paul) -- node[above] {conj} (mary);
    \path (eat) edge[->,newedge,bend right]  node[below] {nsubj} (mary);
  \end{tikzpicture}
  \caption{Relation pointing to the conjuncts}
  \label{fig:pat-nsubj-conj}
\end{subfigure}

\begin{subfigure}{\columnwidth}
  \centering
\begin{tikzpicture}[inner sep=1mm]
  \node[word] (was) {was};
  \node[word] (read) [right=of was] {read};
  \node[word] (watch) [right=of read] {watch};
  \draw[->] (read) -- node[above] {aux} (was);
  \draw[->] (read) -- node[above] {conj} (watch);
  \path (watch) edge[->,newedge,bend left]  node[below] {aux} (was);
\end{tikzpicture}

  \caption{Relation pointing away from the conjuncts.}
  \label{fig:pat-aux-conj}
\end{subfigure}

\begin{subfigure}{\columnwidth}
  \centering
\begin{tikzpicture}[inner sep=1mm]
  \node[word] (house) {house};
  \node[word] (look) [right=of house] {look};
  \node[word] (new) [right=of look] {new};
  \draw[->] (look) -- node[above] {nsubj} (house);
  \draw[->] (look) -- node[above] {xcomp} (new);
  \path (new) edge[->,newedge,bend left]  node[below] {xsubj:nsubj} (house);
\end{tikzpicture}

  \caption{Xcomp special case}
  \label{fig:pat-Xcomp}
\end{subfigure}

\begin{subfigure}{\columnwidth}
  \centering
\begin{tikzpicture}[inner sep=1mm]
  \node[word] (from) {from};
  \node[word] (paris) [right=of from] {Paris};
  \node[word] (come) [right=of paris] {come};
  \draw[->] (paris) -- node[below] {case} (from);
  \draw[->] (come) -- node[below] (lab) {obl} (paris);
  \node at (lab.south) {\textbf{obl:from}};
\end{tikzpicture}
  \caption{Label taken from other word (lemma)}
  \label{fig:pat-obl}
\end{subfigure}

\begin{subfigure}{\columnwidth}
  \centering
  \begin{tikzpicture}[inner sep=1mm]
    \node[word] (boy) {boy};
    \node[word] (live) [right=of boy] {live};
    \node[word] (who) [right=of live] {who};
    \draw[->] (boy) -- node[above] {acl:recl} (live);
    \path  (live) edge[->,newedge,bend right] node[above] {nsubj} (boy);
    \draw[->] (live) -- node[above] {nsubj} (who);
    \path (boy) edge[->,newedge,bend right]  node[below] {ref} (who);
  \end{tikzpicture}
  \caption{Relative clause}
  \label{fig:pat-rel}
\end{subfigure}

\caption{Transformation patterns. Added elements are shown in bold.}
  \label{fig:patterns}
\end{figure}


Adam: figure for this:

John came from Paris






\section{Results}
(ADAM)

\begin{table}[h]
	\centering
	\begin{tabular}{l|rr}
		\textsc{Language} & \textsc{ELAS} & \textsc{EULAS} \\
		\hline
		Arabic &  & \\
		Bulgarian &  & \\
		Czech &  & \\
		Dutch &  & \\
		English &  & \\
		Estonian &  & \\
		Finnish &  & \\
		French &  & \\
		Italian &  & \\
		Latvian &  & \\
		Lithuanian &  & \\
		Polish &  & \\
		Russian &  & \\
		Slovak &  & \\
		Swedish &  & \\
		Tamil &  & \\
		Ukrainian &  & \\
		Average &  & \\
	\end{tabular}
\caption{Test set results}
\end{table}

\begin{table}[h]
	\centering
	\begin{tabular}{l|rr}
		\textsc{Language} & \textsc{ELAS} & \textsc{EULAS} \\
		\hline 
		Arabic &  & \\
		Bulgarian &  & \\
		Czech &  & \\
		Dutch &  & \\
		English &  & \\
		Estonian &  & \\
		Finnish &  & \\
		French &  & \\
		Italian &  & \\
		Latvian &  & \\
		Lithuanian &  & \\
		Polish &  & \\
		Russian &  & \\
		Slovak &  & \\
		Swedish &  & \\
		Tamil &  & \\
		Ukrainian &  & \\
		Average &  & \\ 
	\end{tabular}
	\caption{Gold tree results}
\end{table} 
TODO: Add std dev.

TODO: what is the baseline (just deleting enhanced dependencies from the gold files)?

\section{Discussion/Analysis}
(JP)

It works!

Labels work for english but not well for all languages.

Secondary hyp: EUD do not help for learning EUD.

We do not know about this YET. For this we'd need to use compare a
state of art UD implementation.


To test that we'd need to run our system on the \emph{training} data
of a state of the art EUD system and see how that affects its
performance. But we can't do it since such systems are not available
yet (this is the purpose of the task ...).

Also can be used for bootstrapping.

Also: provide a baseline.

\section*{Acknowledgments}

(Anonymized)

\bibliography{anthology,acl2020}
\bibliographystyle{acl_natbib}


\end{document}
